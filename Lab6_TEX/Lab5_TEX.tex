% указываем класс документа
\documentclass[12pt,a4paper,openany]{extarticle}

% подключаем собственный стилевой файл 
\usepackage{mystyle}

% указываем язык (для автоматической вставки слов, типа "Глава", "Содержание", "Литература", "рис." и пр.
\selectlanguage{russian}

\begin{document}
\part*{Лабораторная работа №5\\
 Математическое описание модели на примере робота Segway}
\section{Методические рекомендации}
До начала работы студент должен выполнить предыдущие работы этого цикла. Необходимо знание основ теоретической механики и математических основ теории систем.
\section{Теоретические сведения}
Мы будем использовать следующие обобщенные координаты:\\
$\theta$: среднее арифметическое значение углов поворота левого и правого колеса,\\
$\psi$: угол наклона робота оносительно вертикали,\\
$\phi$: угол поворота робота относительно начального положения.\\

\begin{equation}
(\theta ,\ \phi)=\left(\cfrac12(\theta_r+\theta_l),\ \frac{R}{W}(\theta_r-\theta_l)\right)
\end{equation}

\begin{equation}
(x_m,\ y_m,\ z_m)=(\int\dot{x}_mdt,\ \int\dot{y}_mdt,\ R),\ 
(\dot{x}_mdt,\ \dot{y}_mdt)=(R\dot{\theta}\cos\phi ,\ R\dot{\theta}\sin\phi)
\end{equation}

\begin{equation}
(x_l,\ y_l,\ z_l)=\left(x_m-\frac{W}{2}\sin\phi ,\ y_m+\frac{W}{2}\cos\phi ,\ z_m\right)
\end{equation}

\begin{equation}
(x_r,\ y_r,\ z_r)=\left(x_m+\frac{W}{2}\sin\phi ,\ y_m-\frac{W}{2}\cos\phi ,\ z_m\right)
\end{equation}

\begin{equation}
(x_b,\ y_b,\ z_b)=\left(x_m+L\sin\psi\cos\phi ,\ y_m+L\sin\psi\sin\phi ,\ z_m+L\cos\psi \right)
\end{equation}

\begin{equation}
(x_b,\ y_b,\ z_b)=\left(\int R\dot{\theta}\cos\phi dt+L\sin\psi\cos\phi ,\ 
\int R\dot{\theta}\sin\phi dt+L\sin\psi\sin\phi ,\ R+L\cos\psi \right)
\end{equation}

Теперь выпишем уравнения поступательной кинетической энергии~$T_k$, вращательной кинетической энергии~$T_p$ и потенциальной энергии~$U$:

\begin{equation}
T_k=\cfrac12 M\left(\dot{x}_b^2+\dot{y}_b^2+\dot{z}_b^2\right)\\
\end{equation}

\begin{equation}
T_p=\cfrac12J_\psi\dot{\psi}^2+\cfrac12J_\phi\dot{\phi}^2
\end{equation}

\begin{equation}
U=Mgz_b
\end{equation}

Если подставить значения соответствующих координат, то мы получим следущее выражение для кинетической энергии: $\cfrac12 M\left(\left(\left(\int R\dot{\theta}\cos\phi dt+L\sin\psi\cos\phi\right)'\right)^2
+\left(\left(\int R\dot{\theta}\sin\phi dt+L\sin\psi\sin\phi\right)'\right)^2
+((R+L\cos\psi)')^2 \right)$. При раскрытии скобок мы получим: $\cfrac12 M\left(2RL\dot\theta \dot\psi\cos\psi+R^2 \dot{\theta}^2+L^2\dot\psi^2+L^2\dot{\phi}^2 \sin^2\psi\right)$\\
Лагранжиан $L$ будет выглядеть следующим образом:

\begin{equation}
L=T_k+T_p-U
\end{equation}

\begin{equation}
L=\cfrac12 M\left(2RL\dot\theta \dot\psi\cos\psi+R^2 \dot{\theta}^2+L^2\dot\psi^2+L^2\dot{\phi}^2 \sin^2\psi\right)+ \cfrac12J_\psi\dot{\psi}^2+\cfrac12J_\phi\dot{\phi}^2-Mg(R+L\cos\psi)
\end{equation}

А уравнение Лагранжа запишем следующим образом:

\begin{equation}
\left\{  
	\begin{array}{rcl} 
	\cfrac{d}{dt}\cfrac{\partial L}{\partial\dot{\theta}}-\cfrac{\partial L}{\partial 			\theta}&=F_\theta\\
	\cfrac{d}{dt}\cfrac{\partial L}{\partial\dot{\psi}}-\cfrac{\partial L}{\partial \psi}&=F_\psi\\
	\cfrac{d}{dt}\cfrac{\partial L}{\partial\dot{\phi}}-\cfrac{\partial L}{\partial \phi}&=F_\phi\\
	\end{array}   
	\right.
\end{equation}

Выполнив указанные дейсвия мы получим:

\begin{equation}
\left\{  
	\begin{array}{rcl}
	F_\theta&=&MR^2\ddot\theta+MRL\ddot\psi\cos\psi-MRL\dot\psi^2\sin\psi\\
	F_\psi&=&MRL\ddot\theta\cos\psi+(ML^2+J_\psi)\ddot\psi-MgL\sin\psi-ML^2\dot\phi^2\sin\psi\cos\psi\\
	F_\phi&=&(ML^2\sin^2\psi+J_\phi)\ddot\phi+2ML^2\dot\psi\dot\phi\sin\psi\cos\psi\\	
	\end{array}   
	\right.
\end{equation}

Далее составим уравнения для двигателей нашего Segway:
\begin{equation}
(F_\theta,\ F_\psi,\ F_\phi)=\left(F_l+F_r,\ F_\psi,\ \frac{W}{2R}(F_r-F_l)\right)
\end{equation}

\begin{equation}
\left\{  
	\begin{array}{rcl}
	F_l&=&k_mI_l\\
	F_r&=&k_mI_r\\
	F_\psi&=&-k_mI_l-k_mI_r\\	
	\end{array}   
	\right.
\end{equation}

\begin{equation}
L\dot{I}_{l,r}=U_{l,r}+k_\omega(\dot\psi-\dot\theta_{l,r})-R_\textit{я}I_{l,r}
\end{equation}
Так как индуктивность обмотки крайне мала, положим $L=0$:
\begin{equation}
I_{l,r}=\frac{U_{l,r}+k_\omega(\dot\psi-\dot\theta_{l,r})}{R_\textit{я}}
\end{equation}
Стоит отметить, что $\phi=\cfrac{W}{R}(\theta_r-\theta_l)$, тогда:
\begin{equation}
\left\{  
	\begin{array}{rcl}
	F_\theta&=&\cfrac{k_m}{R_\textit{я}}(U_l+U_r)+2\cfrac{k_mk_\omega}{R_\textit{я}}(\dot\psi-\dot\theta)\\
	F_\psi&=&-\cfrac{k_m}{R_\textit{я}}(U_l+U_r)-2\cfrac{k_mk_\omega}{R_\textit{я}}(\dot\psi-\dot\theta)\\
	F_\phi&=&\cfrac{W}{2R}\cfrac{k_m}{R_\textit{я}}(U_r-U_l)-\cfrac{W^2}{2R^2}\cfrac{k_mk_\omega}{R_\textit{я}}\dot\phi\\	
	\end{array}   
	\right.
\end{equation}
Запишем окончательную систему уравнений, описывающих динамическую модель робота:
\begin{equation}
\left\{
\begin{array}{rcl}
	\cfrac{k_m}{R_\textit{я}}(U_l+U_r)&=&MR^2\ddot\theta+MRL\ddot\psi\cos\psi-MRL\dot\psi^2\sin\psi+2\cfrac{k_mk_\omega}{R_\textit{я}}(\dot\psi-\dot\theta)\\
	-\cfrac{k_m}{R_\textit{я}}(U_l+U_r)&=&MRL\ddot\theta\cos\psi+(ML^2+J_\psi)\ddot\psi-MgL\sin\psi-ML^2\dot\phi^2\sin\psi\cos\psi-2\cfrac{k_mk_\omega}{R_\textit{я}}(\dot\psi-\dot\theta)\\
	\cfrac{W}{2R}\cfrac{k_m}{R_\textit{я}}(U_r-U_l)&=& (ML^2\sin^2\psi+J_\phi)\ddot\phi+2ML^2\dot\psi\dot\phi\sin\psi\cos\psi-\cfrac{W^2}{2R^2}\cfrac{k_mk_\omega}{R_\textit{я}}\dot\phi\\	
	\end{array}   
	\right.
\end{equation}

Линеарезуем нашу модель. Для этого воспользуемся первым замечательным пределом $ \sin\psi=\psi $, а $ \cos\psi=1 $ при малых углах отклонения, до 15 градусов.

\begin{equation}
\left\{
\begin{array}{rcl}
	\cfrac{k_m}{R_\textit{я}}(U_l+U_r)&=&MR^2\ddot\theta+MRL\ddot\psi-MRL\dot\psi^2\psi+2\cfrac{k_mk_\omega}{R_\textit{я}}(\dot\psi-\dot\theta)\\
	-\cfrac{k_m}{R_\textit{я}}(U_l+U_r)&=&MRL\ddot\theta+(ML^2+J_\psi)\ddot\psi-MgL\psi-ML^2\dot\phi^2\psi-2\cfrac{k_mk_\omega}{R_\textit{я}}(\dot\psi-\dot\theta)\\
	\cfrac{W}{2R}\cfrac{k_m}{R_\textit{я}}(U_r-U_l)&=& (ML^2\psi^2+J_\phi)\ddot\phi+2ML^2\dot\psi\dot\phi\psi-\cfrac{W^2}{2R^2}\cfrac{k_mk_\omega}{R_\textit{я}}\dot\phi\\	
	\end{array}   
	\right.
\end{equation}

Теперь нам необходимо избавиться от оставшихся нелинейностей, квадротов переменных и их произведений между собой. Мы пренебрегаем этими значениями только при условии, что угол отклонения от начального положения будет крайне мал.

\begin{equation}
\left\{
\begin{array}{rcl}
	\cfrac{k_m}{R_\textit{я}}(U_l+U_r)&=&MR^2\ddot\theta+MRL\ddot\psi+2\cfrac{k_mk_\omega}{R_\textit{я}}(\dot\psi-\dot\theta)\\
	-\cfrac{k_m}{R_\textit{я}}(U_l+U_r)&=&MRL\ddot\theta+(ML^2+J_\psi)\ddot\psi-2\cfrac{k_mk_\omega}{R_\textit{я}}(\dot\psi-\dot\theta)-MgL\psi\\
	\cfrac{W}{2R}\cfrac{k_m}{R_\textit{я}}(U_r-U_l)&=& J_\phi\ddot\phi-\cfrac{W^2}{2R^2}\cfrac{k_mk_\omega}{R_\textit{я}}\dot\phi\\	
	\end{array}   
	\right.
\end{equation}

Теперь мы получили систему уравнений, в которой с одной стороны мы имеем управляющее воздействие, а с другой линейные дифференциальные уравнения состояния объекта. В третьем уравнении содержится только одна переменная $\phi$ и ее производные. Поэтому мы будем рассматривать это уравнение отдельно от остальных. Введем вектора состояния $x^T=[\theta, \psi]$ и управления $u^T=[U_l, U_r]$, запишем первые два уравнения в матричной форме:

\begin{equation}
E\ddot{x}+F\dot{x}+Gx=Hu
\end{equation}

\begin{equation}
E=
\begin{bmatrix}
	MR^2&MRL\\
	MRL&ML^2+J_\psi\\
\end{bmatrix},
F=
\begin{bmatrix}
	2\cfrac{k_mk_\omega}{R_\textit{я}}&-2\cfrac{k_mk_\omega}{R_\textit{я}}\\
	-2\cfrac{k_mk_\omega}{R_\textit{я}}&2\cfrac{k_mk_\omega}{R_\textit{я}}\\
\end{bmatrix},
G=
\begin{bmatrix}
	0&0\\
	0&MgL\\
\end{bmatrix},
H=
\begin{bmatrix}
	\cfrac{k_m}{R_\textit{я}}&\cfrac{k_m}{R_\textit{я}}\\
	-\cfrac{k_m}{R_\textit{я}}&-\cfrac{k_m}{R_\textit{я}}\\
\end{bmatrix}.
\end{equation}

Дальше запишем наше уравнение в следующей форме:

\begin{equation}
\ddot{x}=E^{-1}F\dot{x}+E^{-1}Gx+E^{-1}Hu
\end{equation}

Для того чтобы перейти к форме вход--состояние--выход, добавим к имеющемуся уравнению еще одно, $\dot\psi=\dot\psi$. Учитывая, что $E^{-1}G=\cfrac{1}{J_\psi}\begin{bmatrix} 0&M^2RL^2g\\0&M^2R^2Lg\\ \end{bmatrix}$, заменим вектор состояния на $x^T=[\psi, \dot\theta, \dot\psi]$:

\begin{equation}
\dot{x}=\begin{array}{|lcr|} 0&0&1\\ E^{-1}G[1,2]&&\\ E^{-1}G[2,2]&&E^{-1}F \end{array}x+\begin{array}{|lr|} 0&0\\ &E^{-1}H\end{array}u
\end{equation}

\end{document}